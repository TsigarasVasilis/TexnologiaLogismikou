\documentclass[a4paper,10pt]{article}
\usepackage{fontspec}
\usepackage{polyglossia}
\setdefaultlanguage{greek}
\newfontfamily\greekfont{Times New Roman}
\newfontfamily\greekfonttt{Courier New}

\usepackage{graphicx}
\usepackage{float}
\usepackage{tikz}
\usepackage{amsmath}
\usepackage{caption}
\usepackage{hyperref}
\hypersetup{
    colorlinks=true,
    linkcolor=blue,
    filecolor=magenta,      
    urlcolor=blue,
}

\title{Εφαρμογή Streamlit για Ανάλυση scRNA-seq Δεδομένων}
\author{Τσιγάρας Βασίλειος, Καλαμάτας Χρυσόστομος}
\date{}

\begin{document}
\maketitle

\begin{abstract}
Η παρούσα εργασία περιγράφει την ανάπτυξη μίας διαδικτυακής πλατφόρμας για την ανάλυση δεδομένων single-cell RNA-seq (scRNA-seq), βασισμένη στη βιβλιοθήκη Streamlit και το υποσύστημα ανάλυσης Scanpy. Η εφαρμογή επιτρέπει τη μεταφόρτωση αρχείων H5AD, την προεπεξεργασία δεδομένων, τη μείωση διαστάσεων (PCA και UMAP) και την απεικόνιση των αποτελεσμάτων με στόχο την εύκολη χρήση από μη-προγραμματιστές ερευνητές. Περιγράφεται η υλοποίηση, η αρχιτεκτονική, η dockerization και παρουσιάζονται διαγράμματα UML και ενδεικτικές οπτικοποιήσεις.
\end{abstract}

\section{Εισαγωγή}
Η τεχνολογία scRNA-seq επιτρέπει την καταγραφή της γονιδιακής έκφρασης σε επίπεδο μεμονωμένων κυττάρων. Οι αναλύσεις αυτές απαιτούν πολυδιάστατη επεξεργασία και εξειδικευμένες βιοπληροφορικές μεθόδους. Η παρούσα εργασία παρουσιάζει μια απλή web εφαρμογή βασισμένη στο Streamlit, η οποία αυτοματοποιεί τα βασικά βήματα ανάλυσης.

\section{Σχεδιασμός Υλοποίησης}
Η εφαρμογή έχει τρεις κύριες λειτουργικές περιοχές: μεταφόρτωση δεδομένων, προεπεξεργασία και οπτικοποίηση αποτελεσμάτων. Τα δεδομένα φορτώνονται ως αντικείμενα AnnData μέσω Scanpy. Ο χρήστης ρυθμίζει φίλτρα (π.χ. ελάχιστα γονίδια/κύτταρο) και εκτελεί PCA και UMAP. Τα αποτελέσματα αποθηκεύονται και παρουσιάζονται διαδραστικά.

\section{UML Διαγράμματα}

\begin{figure}[H]
\centering
\includegraphics[width=0.85\textwidth]{uml_use_case.png}
\caption{Use Case διάγραμμα: ενέργειες χρήστη στην εφαρμογή.}
\end{figure}

\begin{figure}[H]
\centering
\includegraphics[width=0.85\textwidth]{uml_class_diagram.png}
\caption{Διάγραμμα κλάσεων: βασικά συστατικά του συστήματος.}
\end{figure}

\section{Ανάλυση Υλοποίησης}
Η εφαρμογή είναι γραμμένη σε Python. Το Streamlit διαχειρίζεται το UI με χρήση sidebar και κουμπιών, ενώ η επεξεργασία γίνεται με Scanpy. Ο χρήστης ανεβάζει αρχείο .h5ad, ρυθμίζει φίλτρα και παράγει embedding μέσω PCA/UMAP. Τα δεδομένα αποθηκεύονται σε session state για διατήρηση κατά την πλοήγηση.

\section{Οπτικοποιήσεις Αποτελεσμάτων}

\begin{itemize}
  \item Bar charts για κατανομή batch/cell type
\end{itemize}

\section{Dockerization της Εφαρμογής}

Η εφαρμογή containerized μέσω Docker. Το Dockerfile χρησιμοποιεί ως βάση python:3.10-slim, εγκαθιστά τις απαιτούμενες βιβλιοθήκες από requirements.txt, αντιγράφει τον κώδικα και εκκινεί την εφαρμογή μέσω της εντολής \texttt{streamlit run app.py}. Εκτίθεται η πόρτα 8501 για πρόσβαση από φυλλομετρητή.

\section{Διαθεσιμότητα Κώδικα}
Ο πηγαίος κώδικας της εφαρμογής είναι διαθέσιμος ως έργο ανοιχτού λογισμικού στο GitHub: \url{https://github.com/TsigarasVasilis/TexnologiaLogismikou}. Το αποθετήριο περιλαμβάνει όλα τα αρχεία πηγαίου κώδικα, το Dockerfile για τη δημιουργία container, καθώς και οδηγίες εγκατάστασης και χρήσης.

\section{Συμπεράσματα}
Η εφαρμογή προσφέρει έναν απλό και προσβάσιμο τρόπο ανάλυσης δεδομένων scRNA-seq, χωρίς να απαιτείται εμπειρία στον προγραμματισμό. Η χρήση Streamlit και Scanpy διασφαλίζει ευελιξία και δυνατότητα επέκτασης, ενώ η dockerization επιτρέπει την εύκολη διάθεση σε συνεργάτες ή cloud περιβάλλοντα.

\end{document}
